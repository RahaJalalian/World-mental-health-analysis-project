\documentclass[12pt]{article}
\usepackage{graphicx}
\usepackage{amsmath}
\usepackage{geometry}
\usepackage{hyperref}
\usepackage{titlesec}
\usepackage{tocloft}
\usepackage{float}
\geometry{margin=1in}

\title{Global Mental Health Trends: \\ Analysis of Healthcare Access and Treatment Disparities}
\author{Raha Jalalian, Irmak Aytekin, Diana Borunova, Lihi Galezerl \\
University of Toronto, CSC111 Project 2}
\date{March 2025}

\begin{document}

\maketitle

\begin{abstract}
This report presents a detailed analysis of global mental health data to examine how healthcare availability impacts mental health outcomes, specifically focusing on conditions such as depression, anxiety, and suicide. Leveraging a recursive tree-based data structure, we organize complex, hierarchical statistics across countries and regions. The primary focus of this document is the interpretation and analysis of visual outputs—bar charts, heatmaps, scatter plots, and treemaps—that highlight patterns in mental health service provision, institutional capacity, workforce distribution, and hospitalization metrics.

Through a curated series of nine data visualizations, we explore the extent to which mental health resources (such as beds, psychologists, and occupational therapists) align with population-level mental health needs. Regional disparities emerge clearly in these visualizations, offering critical insight into the mismatch between mental health burden and available infrastructure across continents.

This report does not focus on implementation details or code structure, but rather prioritizes extracting meaningful insights from the graphs and computed statistics generated by the project. By emphasizing data storytelling, we aim to inform future policy decisions, draw attention to underserved populations, and encourage more equitable allocation of mental health resources around the world.
\end{abstract}

\newpage
\tableofcontents
\newpage

\section{Results and Analysis}

\subsection{Depression Rates by Region}

\begin{figure}[H]
    \centering
    \includegraphics[width=0.9\textwidth]{newplot.png}
    \caption{Average Depression Rates per 100K People by Region}
\end{figure}

The chart above reveals significant global disparities in the prevalence of depression. Europe reports the highest average depression rate (5,325.67 per 100K), followed by Oceania and North America. In contrast, Asia exhibits the lowest regional average at 3,672.33 per 100K.

These differences are likely shaped by a mix of factors: actual mental health conditions, cultural stigma, diagnostic capacity, and healthcare access. For example, Europe's high numbers may indicate more open cultural attitudes toward mental illness and better reporting systems, rather than a mental health crisis. Conversely, Asia’s lower numbers might reflect underdiagnosis due to stigma or lack of access to clinical evaluation.

Africa’s average depression rate (4,833.5) is higher than Asia’s despite fewer resources, suggesting that social, historical, and post-colonial stressors may be influencing mental health outcomes. This emphasizes that prevalence data cannot be interpreted in isolation — it must be understood within cultural, economic, and systemic contexts.

Ultimately, these variations call for region-specific strategies to improve diagnosis, access, and support systems, especially in under-resourced areas where mental illness remains underrecognized.

\subsection{Suicide Rate Extremes per Region}

\begin{figure}[H]
    \centering
    \includegraphics[width=0.9\textwidth]{newplot (1).png}
    \caption{Highest vs. Lowest Suicide Rates per 100K People by Region}
\end{figure}

This line chart compares the maximum and minimum suicide rates across regions, exposing stark disparities even within the same geographic areas. Asia shows the most extreme variation, with South Korea reporting a suicide rate above 25 per 100K—the highest in the dataset—while another country in the region falls below 10. This suggests that regional averages can mask deep national differences, and that national policies, stigma, and support systems greatly influence suicide outcomes.

Africa presents the sharpest contrast between the highest and lowest rates: while one country approaches North America's highest levels, another has a near-negligible rate of just over 1 per 100K. This disparity may reflect not only differences in actual suicide occurrence but also underreporting due to stigma, religious taboos, or weak death registration systems.

Interestingly, Oceania has a relatively narrow gap between its highest and lowest values, hinting at more uniform access to care—or, alternatively, a consistent cultural and socioeconomic baseline across the region.

Ultimately, this visualization highlights that regional averages alone are insufficient. Suicide is a complex, multifactorial issue shaped by culture, healthcare access, economic stressors, and social cohesion. Country-level policy intervention and improved data collection are essential to effectively address these disparities.


\subsection{Availability of Mental Health Beds by Region}

\begin{figure}[H]
    \centering
    \includegraphics[width=0.9\textwidth]{newplot (2).png}
    \caption{Average Mental Health Beds per 100K People by Region}
\end{figure}

This chart presents a striking disparity in the availability of mental health beds across regions. Asia leads with nearly 40 beds per 100K people—significantly higher than any other region—despite having the lowest average depression rate. This mismatch raises the possibility that infrastructure exists but is not being effectively leveraged for early diagnosis or treatment, or that those beds are concentrated in a few high-investment countries (e.g., South Korea).

Europe follows with a strong average of nearly 29 beds per 100K, supporting its higher reported prevalence rates with a correspondingly robust treatment capacity. In contrast, Oceania and North America—regions with high depression rates—report alarmingly low bed availability (7.22 and 10.94 respectively), suggesting a potential gap between need and resource allocation.

Africa also remains underserved, with an average of only 11.8 beds per 100K despite notable mental health burdens. South America's numbers are more moderate, hinting at a better-aligned balance between need and infrastructure than in other under-resourced regions.

This visualization underscores the importance of not only having adequate healthcare infrastructure but also distributing it equitably and pairing it with effective outreach, diagnosis, and follow-up care. Mental health capacity cannot be judged solely by prevalence or spending—it must be assessed in terms of alignment between need and available services.


\subsection{Distribution of Psychologists by Region}

\begin{figure}[H]
    \centering
    \includegraphics[width=0.9\textwidth]{newplot (3).png}
    \caption{Psychologists per 100K People by Region}
\end{figure}

This bubble chart displays an imbalance in the global distribution of psychologists. Europe leads with an average of nearly 120 psychologists per 100K people, followed closely by Oceania and South America. In contrast, regions like Asia and Africa report shockingly low numbers—virtually negligible compared to other regions.

The data reveals a pattern that may reflect broader economic and educational disparities. In regions with well-developed healthcare infrastructure, such as Europe and Oceania, the presence of trained mental health professionals is significantly higher. This not only enables better diagnosis and therapy but also promotes early intervention and preventive care. On the flip side, regions like Africa and Asia face both structural and systemic barriers, including fewer training programs, brain drain, and low governmental prioritization of mental health.

Interestingly, North America's count appears modest (around 29 psychologists per 100K), despite its high-income status and public awareness of mental health. This could indicate a mismatch between societal mental health needs and the actual number of licensed professionals available to meet that demand.

Overall, the visualization underscores that the availability of psychologists is a key bottleneck in delivering mental health support. Addressing this imbalance will require not only funding but also educational reform, international collaboration, and destigmatization campaigns that expand both supply and demand for qualified mental health professionals.


\subsection{Bed-to-Depression Ratios per Country}

\begin{figure}[H]
    \centering
    \includegraphics[width=0.95\textwidth]{newplot (4).png}
    \caption{Mental Health Beds to Depression Ratio per Country}
\end{figure}

This scatter plot compares the number of mental health beds to the prevalence of depression in each country, revealing which healthcare systems are more equipped to address mental illness relative to the size of the affected population.

The United States stands out with a notably high bed-to-depression ratio, implying a relatively stronger infrastructure in place to support patients in need of institutional treatment. Other countries like Germany and Argentina also show decent ratios, indicating that they’ve made proportional investments in mental health infrastructure.

However, the majority of countries cluster near the bottom of the graph, indicating extremely low ratios. Notably, India, South Korea, and France all report alarmingly small ratios despite having high depression rates—suggesting a critical mismatch between need and available inpatient resources. This mismatch could reflect policy neglect, hospital overcrowding, or misallocated funding within mental health services.

The wide variation highlights an important insight: raw prevalence data alone cannot measure how well countries support their mentally ill populations. Instead, resource-to-need ratios provide a clearer picture of how well-prepared systems are to handle mental health burdens. Countries with low ratios are especially vulnerable to untreated or inadequately treated cases, which may exacerbate long-term public health issues.

\subsection{Psychologists-to-Suicide Ratios by Country}

\begin{figure}[H]
    \centering
    \includegraphics[width=0.95\textwidth]{newplot (5).png}
    \caption{Psychologists to Suicide Rate Ratios by Country}
\end{figure}

This chart compares each country’s ratio of psychologists per 100K people to suicide rates per 100K—an indirect but powerful proxy for how well-equipped a nation may be to offer preventive care and psychological intervention.

Argentina leads by a wide margin with over 21 psychologists per reported suicide, indicating a deeply embedded mental health infrastructure relative to suicide prevalence. The United Kingdom and Australia also show promising ratios, suggesting a strong emphasis on mental health workforce availability, which could correlate with stronger early-intervention systems and lower barriers to care.

In stark contrast, countries like India, China, and South Korea display ratios approaching zero. Despite having significant populations and documented mental health challenges, these countries show severe shortages in professional psychological support relative to suicide prevalence. This underscores a major systemic gap—one that cannot be bridged by infrastructure alone, but requires public investment in mental health education, training programs, and destigmatization.

The visualization reinforces a crucial insight: a high number of psychologists may not guarantee low suicide rates, but a very low psychologist-to-suicide ratio almost always indicates critical levels of unmet mental health needs.


\subsection{Underserved Countries by Admission Need}

\begin{figure}[H]
    \centering
    \includegraphics[width=0.95\textwidth]{newplot (6).png}
    \caption{Countries with High Mental Health Need vs. Low Hospital Admissions}
\end{figure}

This treemap illustrates the mismatch between mental health need and hospital admissions across countries, measured by the ratio of (depression + anxiety cases) to hospital admissions. Countries with the highest ratios are the most underserved—where need far exceeds capacity.

Egypt emerges as the most underserved country by a significant margin, with an extraordinarily high need-to-admission ratio, suggesting a severe underutilization or outright lack of hospital-based mental health care. South Africa follows closely, pointing to systemic barriers in translating mental health burden into actionable clinical support.

Even high-income countries like the United States, Canada, and Australia appear on this chart with moderate but concerning gaps between need and hospital-based intervention. This suggests that wealth does not guarantee equitable access to care and that other factors—such as insurance systems, urban-rural healthcare distribution, and stigma—may contribute to treatment delays or avoidance.

Conversely, countries like France, Germany, and South Korea display much smaller ratios, indicating stronger alignment between reported mental health burden and institutional response.

This visualization spotlights a critical global issue: untreated or under-treated mental illness is not only a medical challenge, but also a structural one. It calls for governments to examine whether hospital infrastructure, outreach, and accessibility are truly scaled to meet the actual needs of their populations.


\subsection{Total Mental Health Workforce by Region}

\begin{figure}[H]
    \centering
    \includegraphics[width=0.9\textwidth]{newplot (7).png}
    \caption{Total Mental Health Workforce by Region (Psychologists + Occupational Therapists per 100K)}
\end{figure}

This pie chart visualizes the distribution of the total mental health workforce—defined as the combined number of psychologists and occupational therapists per 100K—across different global regions.

Europe and South America clearly dominate, each occupying large slices of the chart. Their combined workforce strength reflects longstanding investments in mental health infrastructure, public health policy, and professional training pipelines. The large share held by South America is particularly striking, revealing a regional commitment that often goes unnoticed in global discourse.

Oceania and North America follow, with moderate workforce sizes relative to population needs. However, given their high depression and anxiety rates, these numbers may still fall short of meeting demand, especially in rural or marginalized areas.

At the extreme low end of the distribution, Asia and Africa contribute barely visible slices. Despite representing large portions of the global population, these regions have the smallest shares of mental health professionals—raising urgent questions about access, capacity, and systemic neglect.

The chart paints a clear picture: without a strong, well-distributed workforce, no amount of infrastructure or funding alone can solve the global mental health crisis. Human capital—trained, accessible, and culturally competent—remains the most vital asset in transforming mental health care systems worldwide.


\subsection{Average Mental Health Admissions in General Hospitals}

\begin{figure}[H]
    \centering
    \includegraphics[width=0.95\textwidth]{newplot (8).png}
    \caption{Average Mental Health Admissions in General Hospitals (per 100K)}
\end{figure}

This heatmap highlights the average rate of mental health admissions in general hospitals across global regions, providing a snapshot of how frequently individuals access institutional care outside of dedicated psychiatric facilities.

Oceania stands out dramatically with the highest rate of general hospital admissions for mental health conditions, suggesting a well-integrated approach where mental health services are embedded within broader healthcare systems. Europe follows with strong numbers, indicating similar mainstream accessibility.

On the other end of the spectrum, Africa reports the lowest admission rates—consistent with earlier visualizations showing workforce shortages, limited infrastructure, and systemic underinvestment in mental health. North America and South America fall into the mid-to-lower range, which is surprising given their relatively high reported prevalence of mental health disorders.

This visualization reveals how general hospitals can either act as accessible entry points to mental health care—or become bottlenecks due to neglect or stigma. High admission rates often reflect not only infrastructure, but also cultural acceptance and clinical integration of mental health into everyday healthcare.

The contrast across regions emphasizes that improving access isn't only about specialized institutions—it’s about building mental health support into all levels of care, especially in systems where stigma or logistical barriers prevent people from seeking help through formal psychiatric channels.


\newpage

\section{Conclusion}

The analysis of global mental health data through a combination of tree-based modeling and visual exploration reveals a deeply uneven landscape in terms of access, capacity, and investment in mental health care. While some regions—particularly Europe and South America—demonstrate relatively strong infrastructure, workforce presence, and hospital admissions, others lag behind significantly, especially Africa and Asia. These disparities are not merely quantitative—they point to systemic gaps that affect real people’s access to care, support, and recovery.

High depression or anxiety prevalence does not always align with adequate treatment infrastructure, as seen in countries with high mental health burdens but low bed-to-depression or need-to-admission ratios. Meanwhile, countries with high ratios of psychologists to suicide rates—like Argentina and the United Kingdom—emerge as examples of proactive investment in prevention and intervention. Still, even high-income countries such as the United States and Canada exhibit critical mismatches between need and service availability, underscoring that economic status alone is not predictive of mental health system effectiveness.

The visualizations not only clarify these mismatches but also emphasize the importance of multidimensional indicators in evaluating mental health care. Looking at depression rates or suicide rates in isolation tells an incomplete story—contextualizing them with resource ratios, workforce availability, and hospitalization data paints a fuller and more actionable picture.

Ultimately, this project underscores the urgent need for targeted, region-specific policies that prioritize equitable access to mental health care. Closing the treatment gap will require not just financial investment, but strategic planning, cultural awareness, and a rethinking of how mental health is integrated into broader healthcare systems.

\newpage
\begin{thebibliography}{9}

\bibitem{ihme}
Institute for Health Metrics and Evaluation. \emph{Prevalence of Mental Disorders by Country}. IHME Global Health Data.\\
\url{https://www.healthdata.org/research-analysis/health-risks-issues/mental-health}. Accessed 5 Mar. 2025.

\bibitem{depression}
World Population Review. \emph{Depression Rates by Country}.\\
\url{https://worldpopulationreview.com/country-rankings/depression-rates-by-country}. Accessed 5 Mar. 2025.

\bibitem{anxiety}
World Population Review. \emph{Anxiety Rates by Country}.\\
\url{https://worldpopulationreview.com/country-rankings/anxiety-rates-by-country}. Accessed 5 Mar. 2025.

\bibitem{suicide}
World Population Review. \emph{Suicide Rates by Country}.\\
\url{https://worldpopulationreview.com/country-rankings/suicide-rate-by-country}. Accessed 5 Mar. 2025.

\bibitem{beds_general}
World Health Organization. \emph{Beds for Mental Health in General Hospitals per Country}. WHO Global Health Observatory.\\
\url{https://apps.who.int/gho/data/node.main.MHBEDS?lang=en}. Accessed 5 Mar. 2025.

\bibitem{beds_mental}
World Health Organization. \emph{Beds in Mental Hospitals per Country}. WHO Global Health Observatory.\\
\url{https://apps.who.int/gho/data/node.main.MHBEDS?lang=en}. Accessed 5 Mar. 2025.

\end{thebibliography}


\end{document}
